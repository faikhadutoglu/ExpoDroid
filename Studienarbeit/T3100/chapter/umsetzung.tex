\chapter{Umsetzung und Ergebnisse}
\label{cha:umsetzung}

\section{Mechanische Konstruktion}
\subsection{CAD-Modellierung}
\subsection{3D-Druck der Komponenten}
\subsection{Montage und Nachbearbeitung}

\section{Elektronische Integration}
\subsection{Schaltungsdesign und Verkabelung}
\subsection{Motoransteuerung und Treiber-Konfiguration}
\subsection{Energieversorgungssystem}
\subsection{Sicherheitsschaltungen}

\section{Softwareentwicklung}
\subsection{Motorsteuerung und Bewegungsalgorithmen}
a
\subsection{Display-Programmierung}
a
\subsection{Animations-Engine für Augen}
a
\subsection{Benutzerschnittstelle}
a
\section{Systemintegration und Test}
a
\subsection{Integrationsstrategie}
a
\subsection{Funktionstests}
\subsection{Fehleranalyse und Optimierung}

\chapter{Ergebnisse und Diskussion}
\section{Funktionsnachweis und Leistungsbewertung}
\section{Anforderungsabgleich}
\section{Lessons Learned}
\section{Kritische Reflexion}



%Je nach Art der Arbeit kann diese Kapitelüberschrift auch \glqq Ergebnisse\grqq~lauten, z.~B. bei rein messtechnischen Aufgaben.
%
%Beschreibung der Umsetzung des zuvor gewählten Vorgehens (theoretische Untersuchung, Erhebungen, Durchführung von Experimenten, Prototypenaufbau, Implementierung eines Prozesses, etc.).
%
%Verifikation anhand der zuvor erarbeiteten Anforderungen und Validierung in Bezug auf das zuvor gestellte Ziel. Diskussion der Ergebnisse. Spätestens hier auch auf die Zuverlässigkeit der gewonnenen Erkenntnisse eingehen (z.~B. anhand der Genauigkeit von Messergebnissen).