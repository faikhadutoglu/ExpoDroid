\chapter{Grundlagen}
\label{cha:Grundlagen}


\section{Mechatronische Systeme}
\subsection{Definition und Systemarchitektur}
\subsection{Interdisziplinäre Integration}

\section{Elektrische Antriebstechnik}
\subsection{Funktionsweise von Elektromotoren (DC/BLDC)}
\subsection{Hoverboard-Motoren und deren Eigenschaften}
\subsection{Motortreiber und PWM-Steuerung}

\section{Mikrocontroller und Steuerungseinheiten}
\subsection{Arduino-Plattform}
\subsection{ESP32 (falls verwendet)}
\subsection{Raspberry Pi (falls verwendet)}
a
\section{Energieversorgung}
a
\subsection{Lithium-Ionen-Akkumulatoren}
a
\subsection{Spannungswandler und Stromverteilung}
a
\subsection{Sicherheitskonzepte (Sicherungen, Battery Management)}
a

\section{Additive Fertigungsverfahren}
a
\subsection{3D-Druck-Technologien}
a
\subsection{Materialien und Nachbearbeitung}
a

\section{Systems Engineering Methodik}
\subsection{V-Modell}
\subsection{Anforderungsmanagement}




%Zielgerichtete theoretische Grundlagen, sowohl fachliche, wie auch methodische.
%
%Zu den Grundlagen gehören \glsentryshort{acr:zb} auch Details zur Problemstellung, der Stand der Technik und weitere Grundlagen, welche zur Konzeptausarbeitung, Umsetzung und Verifikation erforderlich sind.
%
%Grundlagen haben immer einen Bezug zu den nachfolgenden Kapiteln. Diesen Bezug sollte man gelegentlich explizit herstellen, damit bereits in diesem Kapitel klar ist, wo und für was die Grundlagen gebraucht und angewandt werden.